\documentclass[12pt, a4paper, twoside]{article}
\usepackage[T1]{fontenc}
\usepackage[utf8]{inputenc}
\usepackage{amssymb,amsmath}
\usepackage{comment}
\usepackage{datetime}
\usepackage[pdfusetitle]{hyperref}
\usepackage[all]{xy}
\usepackage{graphicx}
\addtolength{\parskip}{.5\baselineskip}

%aqui comeca o que eu fiz de verdade, o resto veio e eu to com medo de tirar
\usepackage{listings} %biblioteca pro codigo
\usepackage{color}    %deixa o codigo colorido bonitinho
\usepackage[landscape, left=1cm, right=1cm, top=1cm, bottom=2cm]{geometry} %pra deixar a margem do jeito que o brasil gosta

\definecolor{gray}{rgb}{0.4, 0.4, 0.4} %cor pros comentarios
%\renewcommand{\footnotesize}{\small} %isso eh pra mudar o tamanho da fonte do codigo
\setlength{\columnseprule}{0.2pt} %barra separando as duas colunas
\setlength{\columnsep}{15pt} %distancia do texto ate a barra

\lstset{ %opcoes pro codigo
breaklines=true,
keywordstyle=\color{blue},
commentstyle=\color{gray},
basicstyle=\footnotesize,
breakatwhitespace=true,
language=C++,
%frame=single, % nao sei se gosto disso ou nao
numbers=none,
rulecolor=\color{black},
showstringspaces=false
stringstyle=\color{blue},
tabsize=4,
basicstyle=\ttfamily\footnotesize % fonte
}

\title{? [UFMG]}
\author{Bruno Monteiro}


\begin{document}
\twocolumn
\date{} %tira a data
\maketitle


\renewcommand{\contentsname}{Índice} %troca o nome do indice para indice
\tableofcontents


%a partir daqui comeca o conteudo mesmo



%%%%%%%%%%%%%%%%%%%%%%%%%
%
% ESTRUTURAS
%
%%%%%%%%%%%%%%%%%%%%%%%%%

\section{Estruturas}

\subsection{BIT}
\lstinputlisting[firstline=3]{Estruturas/BIT.cpp}

\subsection{BIT 2D}
\lstinputlisting[firstline=3]{Estruturas/BIT2D.cpp}

\subsection{Mergesort Tree}
\lstinputlisting[firstline=3]{Estruturas/MergeSortTree.cpp}

\subsection{Order Statistic Set}
\lstinputlisting[firstline=3]{Estruturas/OrderStatisticSet.cpp}

\subsection{SQRT-decomposition}
\lstinputlisting[firstline=3]{Estruturas/SQRT-decomposition.cpp}

\subsection{Seg-Tree}
\lstinputlisting[firstline=3]{Estruturas/SegTree.cpp}

\subsection{Seg-Tree 2D}
\lstinputlisting[firstline=3]{Estruturas/SegTree2D.cpp}

\subsection{Seg-Tree Iterativa}
\lstinputlisting[firstline=3]{Estruturas/SegTreeIterativa.cpp}

\subsection{Sparse-Table}
\lstinputlisting[firstline=3]{Estruturas/SparseTable.cpp}

\subsection{Trie}
\lstinputlisting[firstline=3]{Estruturas/Trie.cpp}

\subsection{Union-Find}
\lstinputlisting[firstline=3]{Estruturas/UnionFind.cpp}



%%%%%%%%%%%%%%%%%%%%%%%%%
%
% GRAFOS
%
%%%%%%%%%%%%%%%%%%%%%%%%%

\section{Grafos}

\subsection{2-SAT}
\lstinputlisting[firstline=3]{Grafos/2SAT.cpp}

\subsection{Bellman-Ford}
\lstinputlisting[firstline=3]{Grafos/BellmanFord.cpp}

\subsection{Floyd-Warshall}
\lstinputlisting[firstline=3]{Grafos/FloydWarshall.cpp}

\subsection{Heavy-Light Decomposition}
\lstinputlisting[firstline=3]{Grafos/HLD.cpp}

\subsection{LCA}
\lstinputlisting[firstline=3]{Grafos/LCA.cpp}

\subsection{LCA com HLD}
\lstinputlisting[firstline=3]{Grafos/LCAcomHLD.cpp}

\subsection{LCA com RMQ}
\lstinputlisting[firstline=3]{Grafos/LCAcomRMQ.cpp}

\subsection{Tree Center}
\lstinputlisting[firstline=3]{Grafos/center.cpp}

\subsection{Centroid decomposition}
\lstinputlisting[firstline=3]{Grafos/centroid.cpp}

\subsection{Dijkstra}
\lstinputlisting[firstline=3]{Grafos/dijkstra.cpp}

\subsection{Dinic}
\lstinputlisting[firstline=3]{Grafos/dinic.cpp}

\subsection{Kosaraju}
\lstinputlisting[firstline=3]{Grafos/kosaraju.cpp}

\subsection{Kruskal}
\lstinputlisting[firstline=3]{Grafos/kruskal.cpp}

\subsection{Ponte}
\lstinputlisting[firstline=3]{Grafos/ponte.cpp}

\subsection{Tarjan}
\lstinputlisting[firstline=3]{Grafos/tarjan.cpp}



%%%%%%%%%%%%%%%%%%%%%%%%%
%
% MATEMATICA
%
%%%%%%%%%%%%%%%%%%%%%%%%%

\section{Matemática}

\subsection{Miller-Rabin}
\lstinputlisting[firstline=3]{Matematica/MillerRabin.cpp}

\subsection{Crivo de Erastosthenes}
\lstinputlisting[firstline=3]{Matematica/crivo.cpp}

\subsection{Exponenciação rápida}
\lstinputlisting[firstline=3]{Matematica/exponenciacaoRapida.cpp}

\subsection{Euclides}
\lstinputlisting[firstline=3]{Matematica/mdc.cpp}

\subsection{Euclides extendido}
\lstinputlisting[firstline=3]{Matematica/mdcExtendido.cpp}

\subsection{Ordem Grupo}
\lstinputlisting[firstline=3]{Matematica/ordemGrupo.cpp}

\subsection{Pollard's Rho}
\lstinputlisting[firstline=3]{Matematica/pollardrho.cpp}

\subsection{Totiente}
\lstinputlisting[firstline=3]{Matematica/totiente.cpp}



%%%%%%%%%%%%%%%%%%%%%%%%%
%
% PROBLEMAS
%
%%%%%%%%%%%%%%%%%%%%%%%%%

\section{Problemas}

\subsection{Inversion Count}
\lstinputlisting[firstline=3]{Problemas/InversionCount.cpp}

\subsection{Area Histograma}
\lstinputlisting[firstline=3]{Problemas/AreaHistograma.cpp}

\subsection{LIS}
\lstinputlisting[firstline=3]{Problemas/lis.cpp}

\subsection{Nim}
\lstinputlisting[firstline=3]{Problemas/nim.cpp}



%%%%%%%%%%%%%%%%%%%%%%%%%
%
% STRINGS
%
%%%%%%%%%%%%%%%%%%%%%%%%%

\section{String}

\subsection{KMP}
\lstinputlisting[firstline=3]{Strings/KMP.cpp}

\subsection{Hash}
\lstinputlisting[]{Strings/hashing.cpp}

\subsection{Z}
\lstinputlisting[firstline=3]{Strings/z.cpp}



%%%%%%%%%%%%%%%%%%%%%%%%%
%
% EXTRA
%
%%%%%%%%%%%%%%%%%%%%%%%%%

\section{Extra}

\subsection{vimrc}
\lstinputlisting{vimrc.txt}

\subsection{Makefile}
\lstinputlisting{makefile.txt}

\subsection{Template}
\lstinputlisting{template.cpp}

\end{document}
